\subsection{DARPA Gran Challenge -- 2005 Driverless Car Competition}

O primeiro vídeo da série apontado \cite{darpa2005} para este relatório é uma matéria de teor jornalístico que acompanhou a competição "Grand Challenge" entre carros autônomos (Driverless Car Competition) de 2005 promovida pelo DARPA (\textit{Defense Advanced Research Projects Agency}, Agência para Projetos de Pesquisa Avançada para Defeasa, em tradução livre, do governo norte-americano).

Neste vídeo, o jornalista descreveu os principais eventos da competição, explicando que a competição envolve uma travessia de 132 milhas (212 quilômetros aproximadamente) no deserto de Mojave, e que o trajeto definitivo só foi revelado às equipes qualificadas duas horas antes do início evento, para fins de lisura da competição.

A competição contou com 23 equipes, sendo os favoritos as equipes: ``Red Team'' com os veículos Highlander e Sandstorm; Stanford Racing Team com Stanley. Highlander pontuou um marco ao ser o primeiro veículo na história a cruzar a marcação de 8 milhas desde as primeiras competições.

O objetivo da competição é ser um grande desafio tecnológico ao demandar que os veículos cruzassem estradas estreitas, leitos de rios e lagos secos, nuvens de poeiras, três tuneis e o trecho sinuoso final "Beer Bottle Pass" próxima a marca de 123 milhas em que os veículos passam por um desfiladeiro.

Por grande parte do trajeto, um ``azarão'' (como nomeou o jornalista), o veículo Dexter da equipe Enco, dominou os tempos, porém devido não a uma falha de sensors ou de software, mas sim por causa de um pneu furado, foi eliminado da competição.

Ao fim da competição, o veículo Volkswagen Tourag, Stanley, da equipe de Stanford foi o primeiro colocado, seguido pelo Sandstorm e Highlander em segundo e terceiro lugares, respectivamente. Além deles, Cat 5 da equipe "Grey" e o Terramax também concluiram o trajeto. Cabe, destaque especial ao Terramax, por ser um veículo de grande porte, indicando uma potencial aplicação militar mais direta considerando a agência patrocinadora do evento.


\subsection{TED Talk - How a driverless car sees the road}

O segundo vídeo proposto \cite{tedtalk} é uma TED Talk apresentado por Chris Urmson em 2015 sobre como carros autônomos veem a estrada. Este formato de palestra, TED Talk, é usualmente adotado como um meio de especialista apresentarem conceitos complexos para o grande público por meio de palestras de curta duração.

O palestrante apresenta uma contextualização acerca do carro, acidentes e vítimas decorrentes desses nos últimos anos (base de referência é 2015, ano da palestra). O palestrante também aborda o desperdício de capital humano com a perda de tempo com trânsito e deslocamento, os quais exigem 100\% de atenção do motorista e impedem que quaisquer outras atividades sejam desempenhadas. Associando esses fatos às tendências de humanos tendem a confiar progressivamente mais em sistemas de auxílio à direção, reduzindo as confiança e atenção dos motoristas. Com base nos pontos elencados, o apresentador depreende que existe uma necessidade contínua de que os sistemas autônomos melhorem para não tornar a vida humana mais confortável, como também para substituir os erros humanos que custam vidas ao redor do mundo todo. 

A partir deste ponto, Chris descreve a maneira pela qual um veículo autônomo observa o ambiente a seu redor, como ele faz a identificação de objetos e com base em todas essas informações, toda decisões de como operar em meio ao trânsito. O apresentador ainda explica que a medida em que os carros autônomos são integrados às cidades e ruas, o desafio tecnológico cresce sensivelmente, exigindo sistemas mais robustos, confiáveis e com maiores velocidades de processamento.

Ao final, o palestrante conclui que o progresso tecnológico tem permitido um otimismo quanto ao futuro do carro autônomo como solução para os diversos problemas não só de trânsito nas grandes cidades, mas como também um elemento mitigador de mortes acidentais decorrentes da imperícia, imprudência ou negligência humanas nas estradas. Apesar de tudo, o apresentador reforça que espera ver seus filhos não necessitando de carteira de motorista, em mundo em que o carro autônomo é a norma.

\subsection{Audi A8 2018}

Distintamente dos vídeos anteriores, o terceiro vídeo \cite{audiA8} trás uma avaliação técnico-comercial do, a época, novo Audi A8 cujas especificações técnicas apontam para um veículo com capacidade de operação autônoma nível 3 (segundo narrador do vídeo). A avaliação feita conversa com os dois vídeos anteriores, em que no primeiro \cite{darpa2005} uma competição de carros em um nível acadêmico buscava estabelecer as melhores tecnologias candidatas, dez anos mais tarde Chris Urmson \cite{tedtalk} propõe a necessidade preemente dos veículos autônomos como futuro e que ainda haveria um tempo até que eles estivessem disponíveis no mercado. A apresentação, três anos depois da TED Talk sobre carros autônomos, com um modelo comercial -- ainda que de luxo --  revela um progresso tecnológico sensível.

A avaliação técnica do vídeo dedica a primeira metade do vídeo a discutir a tecnologia embarcada no modelo A8, descrevendo centrais de processamento de dados, sensores, seus tipos e funções e como todos colaboram para as funções autônomas. O maior destaque é a capacidade de o Audi A8 operar independentemente, se assim seu motorista escolher, em situações de trânsito pesado cujo contexto é uma sequência de ``andar e parar'' onde o motorista arranca o veículo e logo em seguida para em função do trânsito, semáforos ou outras eventualidades. Além disso, o narrador destaca a capacidade que o modelo possui em identificar marcações visuais, como ``guard-rails'' e linhas tracejadas que denotam a separação entre pistas.

Como outra medidade de facilidade, o Audi A8 também dispõe de um sistema de inteligência para estacionamento e manobras em garagem, reduzindo riscos de colisões ou manobras inadequadas em espaços usualmente limitadas, caracterizados pela baixa acuidade visual dos motoristas nestes ambientes.

O narrador ainda ressalta que embora o modelo disponha do sistema de navegação autônoma, a legislação de cada país não era clara o suficiente para permitir ou não a adoção do sistema, nos casos em que os países proibiam ou limitavam o uso de tais tecnologias, o modelo vinha com os sistemas desabilitados.

\subsection{Tesla supercomputer}

O quarto vídeo da série proposta \cite{dojo} discute como a Tesla emprega supercomputadores para treinar sistemas de direção autônomos usando redes neurais como uma estratégia de mercado a fim de se tornar independente de empresas terceiras, como Google e Nvidia.

Ao longo do vídeo são apresentados detalhes técnicos do supercomputador Dojo e as principais vantagens de adotar tal sistema, incluindo a redução da dependência de hardware da Nvidia e de soluções da Google no desenvolvimento de veículos autônomos. O apresentador concentra-se em descrever como esses carros operam matematicamente, com destaque ao conceito de matrizes esparsas, e explica que os chips desenvolvidos pela Tesla são fisicamente otimizados para lidar com esse tipo de estrutura.

Com a verticalização da tecnologia de direção autônoma, a Tesla, segundo o autor do vídeo, consolida-se como forte candidata a lançar um dos primeiros veículos plenamente operacionais em ambiente urbano. O supercomputador não apenas resolve problemas matemáticos com maior eficiência, mas também reduz significativamente o tempo de treinamento. Além disso, o uso de redes neurais não supervisionadas permite que os próprios algoritmos identifiquem e classifiquem placas, eventos e intercorrências de forma autônoma, sem intervenção humana, permitindo com que processo de aprendizado se torne ainda mais rápido e escalável.
