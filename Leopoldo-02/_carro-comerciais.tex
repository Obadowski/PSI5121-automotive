\subsection{Mercedes-Benz Drive Pilot}

O sistema DRIVE PILOT (nome comercial definido pela montadora em seu site oficial) \parencite{mercedes2023drivepilot} é considerado um item opcional que pode ser adicionado aos modelos de luxo S-class e EQS, o primeiro de motorização híbrida e o segundo totalmente elétrico, e opera em conformidade com as especificaçãoes SAE-3. Segundo a fabricante, é esperado que o veículo seja capaz de assumir a direção por velocidade até 95 km/h, a montadora alerta em sua documentação explicitamente que o veículo apesar de ser capaz de operar de forma independente ainda requer um motorista. Tecnicamente, a arquitetura do DRIVE PILOT consiste no emprego de redundâncias duplas nos sistemas elétricos, de manobra e frenagem com dois níveis de \textit{fallback} em caso de falha, em um primeiro momento transfere o comando de volta ao motorista para que tome as devidas providências, porém se isso não ocorrer em um tempo determinado, o sistema opera buscando a parada do veículo com acionamento das luzes de emergência.

\subsection{BMW Personal Pilot L3}

Sistema concorrente ao apresentado na subseção anterior, a tecnologia da BMW opera também em seus veículos de luxo da Série 7 \parencite{bmw2023personalpilot} também no nível SAE-3 em que o motorista pode deixar a operação por conta do veículos (comercialmente falando, o texto refere-se a tirar as mãos do volante e os olhos da estrada). A versão desta fabricante opera com velocidades de até 60 km/h, também emprega sensores redundantes (radares, câmeras e um sistema LiDAR) além de mapas e sistema de comunicação 5G para atualizações em tempo real. Segundo informado no texto comercial, depreende-se que o veículo empregue mapas tridimensionais das regiões com atualizações dependentes da conexão à internet disponível pelo próprio veículo (conectividade 5G), além disso o serviço também é um opcional oferecido pela fabricante aos clientes.

\subsection{Honda SENSING Elite}

O sistema da fabricante japonesa \parencite{honda2021sensingelite} chamado de SENSING Elite é oferecido no modelo Honda Legend Ex apenas no japão e representa o ingresso de veículo autônomos SAE-3 no mercado japonês. Um dos recursos de destaque que o modelo oferecido é a operação \textit{Traffic Jam Pilot} em que o sistema opera o veículo em velocidades de até 30 km/h, adequado para condução durante congestionamentos (um problema frequente em grandes cidades, especialmente no Japão com altíssima densidade demográfica). A tecnologia combina sensores, câmeras e sistemas LiDAR em complemento a mapas tridimensionais combinados com sistemas de posicionamento (tecnologia similar ao GPS) possibilitando uma operação segura e adequada sem a atenção do motorista. O informativo da empresa ainda destaca que o veículo retorna o comando ao motorista em caso de ``melhora do tráfego'', isto é, saída da condição da condição de congestionamento com aumento da velocidade acima de 30 km/h.

\subsection{Percepção global}

Embora os veículos tenham apresentações temporais distintas, o modelo da Honda é de 2021, BMW do fim de 2023 e da Mercedez Benz em 2024 é perceptível o incremento de velocidade na operação do veículo autônomo. Pois, a tecnologia japonesa focada em congestionamentos limitada a 30 km/h é superada em dois anos pela versão alemã direcionado à condução nas Autoban (estradas de alta velocidade) para velocidades de 60 km/h e pouco mais de um ano, um incremento de quase 50\% pela Mercedes cujo veículos operam até 95 km/h de forma autônoma. Com isto, percebe-se que há uma progressão evidente do mercado automotivo em direção às operações autônomas mais gerais dos veículos, inicialmente focados no ambiente urbano e, agora, com resultados relevantes na direção em autoestradas.