\subsection{Tesla vs Waymo}

O vídeo \parencite{cleanerwatt} compara as abordagens da Tesla e da Waymo como soluções para direção autônoma de veículos terrestres. Ao longo do vídeo, o autor apresenta pontos de vista das duas empresas, inclusive com trechos do discurso do CEO da Tesla acerca das capacidades computacionais de seus veículos e quais vantagens ele teria. O argumento central do autor é a comparação entre o sistema LiDAR (\textit{Light Detection And Ranging}) e a visão computacional desenvolvida pela Tesla. Há, portanto, uma conclusão necessária de que dados se tornam fundamentais para ambas as estratégias. Na primeira, da Waymo, os dados são usados para o reconhecimento prévio dos ambientes, que, somados às capacidades de monitoramento em tempo real dos sensores, permitiria a operação de forma autônoma de veículos, mas apenas dentro de territórios pré-mapeados e geograficamente limitados (geofencing). Em contrapartida, a abordagem da Tesla, que demanda alto grau de treinamento por meio de Deep Learning, é projetada para ser universalmente escalável. A tese do vídeo é que a gigantesca frota de veículos da Tesla funciona como uma rede de coleta de dados em escala global, o que lhe confere uma vantagem decisiva a longo prazo para criar uma solução de autonomia generalizável e vitoriosa.

\subsection{Stack tecnológica de veículos autônomos}

No segundo vídeo proposto \parencite{frankyoutube}, o autor apresenta a importância da chamada \textit{stack} técnica, ou ``pilha de conceitos'' técnicos, necessária para experimentação de um veículo autônomo. Durante a apresentação, há o comparativo entre duas abordagens tecnológicas distintas, a primeira da Ford com um grande conjunto de sensores e processamento de dados com base nas informações coletadas em tempo real e a segunda avalia a proposta da Tesla cuja solução emprega um número menor de sensores, porém aposta em uma unidade de processamento baseada em inteligência artificial. O contraste central entre as estratégias revela mundos distintos, o primeiro com mais sensores, maior qualidade de dados ambientais e significativa redundância de sensores contrabalança essa visão o maior custo, necessidade de equipamentos e tecnologias mais sofisticados mesmo para trajetos de deslocamento simples. Ao concluir as comparações, o autor do vídeo expõe as diferenças em que a Ford entende sua tecnologia LiDAR como um elemento de segurança indispensável ao passo que a Tesla aposta na sofisticação de seus algoritmos como compensação.

\subsection{Papel do Deep Learning em veículos Autônomos}

No vídeo do canal BASIC Thinking \parencite{basicthinking}, o tópico se afasta da competição mercadológica apresentada nos vídeos anteriores, para os conceitos fundamentai da inteligência artificial que subsidiam, ou teoricamente habilitam, a autonomia veicular. O centro de seu argumento é o sistema de condução da Audi, que emprega \textit{Deep Learning}, que segundo o apresentador do vídeo até um passado recente não haviam dados nem capacidade de processamento suficiente para analisar dados a ponto de permitir que sistemas de aprendizado profundo operassem veículos de forma autônoma. As capacidades são demonstradas por meio de estacionamento automático por meio de compreensão do seu contexto, o que serve como argumento base para a conclusão do vídeo de que atingir autonomia verdadeira, o veículo deve ser capaz não apenas de analisar, mas prever e reagir rapidamente, de forma segura e natural, aos diferentes desafios que a condução no mundo real apresenta.
