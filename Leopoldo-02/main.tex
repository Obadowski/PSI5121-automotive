\documentclass[12pt, a4paper]{article}

\usepackage{fullpage}
\usepackage[utf8]{inputenc}
\usepackage[brazilian]{babel}
\usepackage{url}

\usepackage[style=abnt,backend=biber,language=portuguese]{biblatex}
\addbibresource{refs.bib} % ou o nome real do seu .bib

\DeclareFieldFormat{url}{\texttt{\url{#1}}}
\DeclareFieldFormat{urldate}{Acesso em:~#1}
\renewcommand*{\biburlsetup}{\Urlmuskip=0mu plus 1mu\relax}
\urlstyle{tt}

\usepackage{lipsum}

\title{Relatório 01 -- DARPA Challenge}
\author{Vinícius Obadowski}
\date{Julho 2025}

\usepackage[hidelinks]{hyperref}
        
\begin{document}

\maketitle

\section{Introdução}

Este relatório apresenta breves comentários sobre três vídeos \cite{cleanerwatt,basicthinking,frankyoutube} sobre tecnologia de carros autônomos e suas perspectivas futuras. Além disso, na segunda parte é apresentada uma avaliação do artigo \cite{Pendleton2017Autonomous} sobre percepção, planejamento, controle e coordenação de veículos autônomos.

\section{Vídeos}

\subsection{Tesla vs Waymo}

Discute o primeiro vídeo \cite{cleanerwatt}

\subsection{Stack tecnológica de veículos autônomos}

Fala do segundo vídeo \cite{frankyoutube}

\subsection{Papel do Deep Learning em veículos Autônomos}

Terceiro vídeo \cite{basicthinking}


\section{Percepção, planejamento, controle e coordenação para veículos autônomos}

\subsection{Introdução}

Falarei apenas do \parencite{Pendleton2017Autonomous}.

\section{Veículos L3 comerciais}

\subsection{Modelo 1}

\lipsum[1]

\subsection{Modelo 2}

\lipsum[1]

\subsection{Modelo 3}

\lipsum[1]

% \bibliographystyle{plain}  % ou plain, alpha, etc.
% \bibliography{refs}
\printbibliography


\end{document}

